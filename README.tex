\startchapter[title={uREST}]

\startsection[title={Background}]

This library is designed to enable simple API's to be built on
micro-controllers, based on a sub-set of the REST API design principles,
and inspired by the design of the \goto{Apollo
DSKY}[url(https://history-computer.com/apollo-guidance-computer/)]
guidance computer. Rather than build a full HTTP server stack, including
JSON parser, and supporting the full complexity of modern REST API's,
this library aims to support simple operations in a resource constrained
environment.

Like the DSKY unit, it is assumed that all the \quote{objects}
representing the states we are interested in are held in
\quote{\goto{nouns}[url(https://dlove24.github.io/urest/urest/api/base.html)]}.
The HTTP actions then represent \quote{verbs} which dictate the actions
on the noun. So each API call is then in the form of \quote{verb-noun};
e.g.~\quote{GET /led}, or \quote{PUT /led}. Valid verb actions are

\startplacetable[location=none]
\startxtable
\startxtablehead[head]
\startxrowgroup[lastrow]
\startxrow
\startxcell[width={0.04\textwidth}] Verb \stopxcell
\startxcell[width={0.07\textwidth}] HTTP Method \stopxcell
\startxcell[width={0.89\textwidth}] Action \stopxcell
\stopxrow
\stopxrowgroup
\stopxtablehead
\startxtablebody[body]
\startxrow
\startxcell[width={0.04\textwidth}] Get \stopxcell
\startxcell[width={0.07\textwidth}] \type{GET} \stopxcell
\startxcell[width={0.89\textwidth}] Return the current state of the
requested noun. \stopxcell
\stopxrow
\startxrow
\startxcell[width={0.04\textwidth}] Set \stopxcell
\startxcell[width={0.07\textwidth}] \type{PUT} \stopxcell
\startxcell[width={0.89\textwidth}] Set the requested noun to {\em
exactly} the specified state. This is assumed to be idempotent, with the
resultant state matching exactly the request from the client. \stopxcell
\stopxrow
\startxrow
\startxcell[width={0.04\textwidth}] Update \stopxcell
\startxcell[width={0.07\textwidth}] \type{POST} \stopxcell
\startxcell[width={0.89\textwidth}] Update the state requested noun.
This is {\em not} assumed to be idempotent: for instance asking a noun
to move between two states on each update. \stopxcell
\stopxrow
\startxrowgroup[lastrow]
\startxrow
\startxcell[width={0.04\textwidth}] Delete \stopxcell
\startxcell[width={0.07\textwidth}] \type{DELETE} \stopxcell
\startxcell[width={0.89\textwidth}] Remove the current state of the
noun, and return to a the default state. This does {\em not} remove the
noun from the API: only the state currently held by the API. \stopxcell
\stopxrow
\stopxrowgroup
\stopxtablebody
\startxtablefoot[foot]
\stopxtablefoot
\stopxtable
\stopplacetable

In all cases the body of the HTTP request is a simple collection of
\quote{key: value} pairs, formatted as a
\goto{JSON}[url(https://www.json.org/json-en.html)] object. Only a
sub-set of the JSON specification is used: in particular multiple
objects are not allowed, and nor are arrays (i.e.~\quote{\type{[]}}) of
any sort. This both simplifies the parsing, and especially the memory
required for the parser, and reinforces the intent to support only
minimal API's.

\stopsection

\startsection[title={Installation}]

A package of this library is provided on PyPi as
\goto{\type{urest-mp}}[url(https://pypi.org/project/urest-mp/)]. This
can be installed with the normal Python tools, and should also install
to boards runnning MicroPython under
\goto{Thonny}[url(https://thonny.org/)].

For manual installation, everything under the \type{urest} directory
should be copied to the appropriate directory on the MicroPython board,
usually \type{/lib}. The library can then be imported as normal, e.g.

\type{python from urest.http import RESTServer from urest.api import APIBase}

See the documentation for the
\goto{examples}[url(https://dlove24.github.io/urest/urest/examples/index.html)]
for more detailed guidance on the use of the library. This code is also
available in the \type{urest/examples} folder, or as the library
\type{urest.examples} when the package is installed.

\stopsection

\startsection[title={Debugging}]

Console output from the \type{urest.http.server.RESTServer} is
controlled by the standard \type{__debug__} flag. By default no output
will be sent to the \quote{console} {\em unless} the \type{__debug__}
flag is \type{True}.

{\bf Note:} that in the standard Python environments the status of the
\type{__debug__} flag is often controlled by the optimisation level of
the interpreter: see the standard \goto{Python
documentation}[url(https://docs.python.org/3/using/cmdline.html\#cmdoption-O)]
for more details. For MicroPython the status of the \type{__debug__}
flag is set by \goto{internal
constants}[url(https://docs.micropython.org/en/latest/library/micropython.html\#micropython.opt_level)].
However if the \type{__debug__} constant is set whilst a programming is
running the \goto{results may be
unexpected}[url(https://forum.micropython.org/viewtopic.php?t=6839)],
due to optimisations undertaken by the MicroPython lexer. Instead for
MicroPython set the status of the \type{__debug__} flag in the platform
standard \type{boot.py} or similar: see the documentation for the
specific port for more details.

\stopsection

\startsection[title={Design}]

The core of the library is a simple HTTP server, specialised to the
delivery of a REST-like API instead of a general HTTP server. The
design, and the use of the \type{asyncio} library, is inspired by the
\goto{MicroPython HTTP
Server}[url(https://github.com/erikdelange/MicroPython-HTTP-Server)] by
Erik de Lange. This library uses a roughly similar structure for the
core of the \type{asyncio} event loop, and especially in the design of
the
\goto{\type{RESTServer}}[url(https://dlove24.github.io/urest/urest/http/server.html)]
class.

Key differences include

\startitemize
\item
  Support for \type{PUT}, \type{POST} and \type{DELETE} operations, in
  addition to \type{GET}. These are required for an API server, and also
  form the \quote{verbs} of the actions allowed on the \quote{nouns} by
  the API built on-top of this library.
\item
  A more object-oriented design of the call/response handler, made
  easier this library is {\em not} a general HTTP server. For instance
  Python \type{getters} and \type{setters} are used where possible for
  input validation, and the central API response is based on the
  \goto{\type{APIBase}}[url(https://dlove24.github.io/urest/urest/api/base.html)]
  abstract base class
\item
  A more explicit validation of input from the network layer, especially
  in the assumption that all input is by default hostile. This library
  should serve as an example of best-practice in protocol handling; at
  least in the slightly resource constrained environment of MicroPython
\item
  This implementation is principally a teaching library, so the
  \goto{Documentation}[url(https://dlove24.github.io/urest/urest)]
  should be at least as important as the \quote{code}. Where possible
  all algorithms and implementation techniques should also be explained
  as fully as possible, or at least linked to reference
  standards/implementations
\item
  For consistency, all code should also be in the format standardised by
  the \goto{Black}[url(https://github.com/psf/black)] library. This
  makes it easier to co-ordinate external code and documentation with
  the implementation documented here.
\stopitemize

\stopsection

\startsection[title={Known Implementations}]

\startitemize[packed]
\item
  Raspberry Pi Pico W (MicroPython 3.4)
\stopitemize

\stopsection

\stopchapter
